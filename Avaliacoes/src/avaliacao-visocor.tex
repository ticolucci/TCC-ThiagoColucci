\documentclass[12pt,a4paper]{article}
\usepackage[brazil]{babel}
\usepackage[utf8]{inputenc}
\usepackage[T1]{fontenc}

\usepackage{syntonly}

\title{Avaliação de TCC\\Visocor}
\author{Thiago Rodrigues Colucci}

\pagestyle{headings}

%\syntaxonly

\usepackage[pdftex,bookmarks,colorlinks]{hyperref}

\begin{document}
\maketitle

\newpage

\tableofcontents
\newpage
\section{Dados do TCC}
\textbf{Aluno:}\\
Fernando Raganham Barbosa

\textbf{Ano em que cursou a disciplina:}\\2009


\textbf{Nota obtida:}\\5.0


\section{Resumo da monografia}
O texto constitui-se, basicamente, da explicação de como armazenar sinais no computador, depois o que são ruídos de sinal, seguido de uma explicação sobre filtros para sinais e finalmente uma vasta explicação sobre como chegar na fórmula de tais filtros e como compará-los.

Por se tratar de um texto científico sobre filtros de som, a monografia é repleta de fórmulas que explicam como chegar na equação final de redução de ruídos, no caso estudado, constantes e aditivos. Depois de explicados alguns tipos de filtros de Wiener e como compará-los, o aluno explicou como produziu o experimento de comparação entre 4 tipos de implementações de tal filtro e mostrou os resultados. Conclusão: o melhor filtro é o Wiener LMS adaptativo, pois é o de menor custo computacional com a maior redução do ruído testado e menor distorção do sinal original.

\section{Avaliação}
\subsection{Parte Técnica}
O texto é claro, porém existem trechos que não seguem uma lógica ``construtiva'', tornando-os repetitivos e difíceis de entender.

No geral, conseguiu abordar todos os pontos propostos. Porém muitos deles ficaram apenas ``jogados'', não dando continuidade nem mostrando onde foram utilizados. Além disso, existem alguns termos que são citados e não possuem referência, nem explicação, o que torna a leitura muito difícil para leigos no assunto (como eu).

\subsection{Parte Subjetiva}
De fato, o aluno mostrou várias matérias que influenciaram o trabalho dele durante sua graduação. No entanto, não contou experiências na graduação. Outro ponto que me estranhou foi a parte ``Dificuldades e Frustrações'', pois esperava algo relacionado ao curso, porém ele só mencionou frustrações em relação ao trabalho. Agora estou com dúvida sobre o que é esperado nesta seção.

\section{Comentários}
Escolhi este trabalho por ser o de menor nota dentre os do ano passado, assim pude aprender com os erros do meu colega veterano que dificultaram a leitura, para tentar não cometer na minha monografia. Tais como:
\begin{itemize}
\item parágrafos repetitivos;
\item termos sem explicações ou referências;
\item citar explicações em outras seções usando um sinônimo (muitas vezes fiquei perdido no texto);
\end{itemize}
Mesmo assim o trabalho me pareceu bom. Portanto terei que me esforçar muito para tirar o sonhado 10.

\end{document}