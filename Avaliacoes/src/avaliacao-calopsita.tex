\documentclass[12pt,a4paper]{article}
\usepackage[brazil]{babel}
\usepackage[utf8]{inputenc}
\usepackage[T1]{fontenc}

\usepackage{syntonly}

\title{Avaliação de TCC\\Calopsita}
\author{Thiago Rodrigues Colucci}

\pagestyle{headings}

%\syntaxonly

\usepackage[pdftex,bookmarks,colorlinks]{hyperref}

\begin{document}
\maketitle

\newpage

\tableofcontents
\newpage
\section{Dados do TCC}
\bf{Alunos:}\\
Cauê Haucke Porta Guerra\\
Cecilia Fernandes\\
Lucas Cavalcanti dos Santos\\


\bf{Ano em que cursaram a disciplina:}\\2009


\bf{Nota obtida:}\\10.0


\section{Resumo da monografia}

O Calopsita é um sistema gerenciador de projetos ágeis. Já existem alguns outros projetos semelhantes, porém o Calopsita possui duas grandes diferenças: uma interface humano computador excelente e customização do projeto muito boa. Com esses recursos e sendo open source, este sistema tem tudo para ganhar muito mercado nos próximos anos. Eu mesmo já comecei a utilizá-lo me pequenos projetos e pude, desta forma, prover feedback para a atual equipe de Laboratório de XP.

Durante a monografia, os alunos explicaram quais metodologias utilizaram no desenvolvimento do sistema. Basicamente XP com algumas boas práticas de programação, como BDD e DDD. A Monografia foi basicamente dividida nestas duas partes: explicação do sistema e explicação das metodologias.

\section{Avaliação}
\subsection{Parte Técnica}
O texto está muito bem escrito, claro e sem erros. Quando acabei de ler percebi que a monografia abordou e explicou mais temas e tópicos do que esperava. Todos os tópicos de código foram exaustivamente explicados e bem descritos. Apenas achei que alguém sem conhecimento nenhum sobre métodos ágeis e as ``boas práticas'' não conseguiria ler a monografia, mas, como o texto possui bastante referências, um leitor leigo poderia aprender o necessário através dessas referências.

\subsection{Parte Subjetiva}
Basicamente todos os alunos contaram como chegaram ao IME, como foi essa etapa da vida e o que esperam do futuro. Depois citaram algumas disciplinas relevantes para o projeto, mas principalmente relevantes para a formação como profissionais da área. Destaco agora a Cecilia por ter sido a única que realmente apontou falhas concretas na graduação, enquanto os outros disseram matérias ruins ou frustrações com o projeto.

\section{Comentários}
Não tenho comentários a fazer, apenas desejo que minha monografia fique tão clara e objetiva quanto a do Calopsita.
\end{document}