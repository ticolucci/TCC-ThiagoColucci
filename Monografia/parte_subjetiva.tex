\documentclass[11pt,a4paper]{article}
\usepackage[brazil]{babel}
\usepackage[utf8]{inputenc}
\usepackage[T1]{fontenc}

\usepackage{syntonly}

\usepackage{anysize}
\marginsize{3cm}{2cm}{2cm}{1cm}


\title{Parte Subjetiva do TCC}
\author{Thiago Rodrigues Colucci\\\\Orientador: Fabio Kon}


\pagestyle{headings}


\begin{document}
\maketitle

\newpage

\tableofcontents
\newpage
\textbf{Aluno:} Thiago Rodrigues Colucci

\textbf{Orientador:} Prof. Dr. Fabio Kon


Após trabalhar numa empresa, percebi que queria algo a mais do que ``simplesmente'' trabalhar em uma empresa de consultoria e desenvolvimento. Então no início deste ano (2010) resolvi procurar algum projeto para mestrado ou doutorado. Conversei com o Prof. Fabio Kon, este tinha dois projetos sobre composições de serviços web, nunca tinha estudado nada sobre o assunto. Li um artigo e procurei mais um pouco na internet e gostei do tema. Entrei para o projeto CHOReOS, e consequentemente para o projeto Baile, e a escolha do tema para o TCC foi conseqüência de um dos \emph{deliverables} do projeto CHOReOS.



\section{ Desafios e frustrações encontradas}





\section{Lista das disciplinas cursadas no BCC mais relevantes para o trabalho}







\section{Observações sobre a aplicação de conceitos estudados nas disciplinas do curso}






\section{Futuro do projeto}

\end{document}