\section{Conclusions}


At the end of the development phase we concluded that, with the test suite the TDD would have created, some of the refactorings would have been easier to do and we could have avoided a lot of small defects, such as forgetting to include a line in the server property file after one big refactoring. But creating the orchestrations from scratch was probably the most difficult part of the development of this work. If we had a test framework and development methodology for the orchestrations development, we probably would not have spent the time we did trying to deploy simple orchestrations. Such a framework does not exists yet, this is another goal of the CHOReOS project.

The biggest contribution of this work, was to create the software that synthesize orchestrations. The Ruby script has some interesting \emph{Domaing Specific Languages} (DSL), as shown in snippets \ref{each-node} and  \ref{each-node-parallel}. The later hides, from the user, all the details of creating, managing, and joining the Process or Threads (according to the Ruby's version, 1.8 or 1.9). With this kind of helper methods, its easier to write the orchestration files (\emph{BPEL}s, \emph{WSDL}s, and \emph{JBI} descriptors) that will be the foundations of the generated composition.

\lstinputlisting[caption=Example of \emph{DSL} (\textbf{each\_node}) from Ruby script ``generate_orchestration'' ,language=Ruby,label=each-node]{images/each-node.rb}
\lstinputlisting[caption=Example of \emph{DSL} (\textbf{each\_node\_parallel}) from Ruby script ``generate_orchestration'' ,language=Ruby,label=each-node-parallel]{images/each-node-parallel.rb}

 
After the statistical analysis of the experiments was done, we concluded:
\begin{enumerate}
	\item All three topologies scales linearly according to the frequency of submission. For example, with fixed messages of 1MB, when we submitted the balanced tree to a frequency of 10 messages per second, it took an average of 186.98 seconds for the response; with a frequency of 100 messages per second the system took about 10 times more to respond (1876.71 seconds)
	\item All three topologies are more influenced by the size of messages than by the frequency. The case with the biggest difference was the ``Horizontal Tree'', where the size influences $45.51\%$ of the total \emph{variation}, while the frequency influence is only $27.73\%$.
	\item In theory, if there was any significant difference between the average response times, it would be that the ``Vertical Tree'' should be slightly fast than the ``Horizontal Tree'', with ``Balanced Tree'' as the mean case. This difference would occur because of possibles overheads on the interpretation of PEtALS services orchestration and also because of the JVM's JIT Compiler, that would optimize the ``Nodes'', but not the ``Leaves'' (because the ``Node'' service is still active while waiting for its children reply, while the ``Leaf'' process has to be awoken from Memory every time it receives a message). The tests confirmed the suspicions that the ``Vertical Tree'' was slightly faster than the ``Horizontal Tree'', but we did not expect that the ``Balanced Tree'' had the fastest execution of all. This specific case will be deeply studied later.
\end{enumerate}


As future work, we intend to  create new test cases, with more nodes, different topologies, and parallel children invocation. We will also study in more detail the ``Balanced Tree'' case, and to that end we will create more experiments with more values for each parameter of ``send_messages'' script. Moreover, with the ``framework'' and the future tests, the understanding of the ``Internet of the Future'''s scalability will be much easier to do.