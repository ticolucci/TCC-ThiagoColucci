\section{Introduction}

With the development of the Internet, we have been able to use a lot of Web Applications (as ``Gmail'', ``Google Docs'' and ``Amazon.com''). Those applications use a lot of web services for decoupling it's parts. Now that we have plenty of services to use, we need some technic to compose those web services into new ones with higher level of abstraction. In this context arose the Orchestration and the Choreography of Web services, technics to create services using only description languages and pre-existing web services.

Business process built using compositions become interoperable and scalable, by the fact that they depend basically upon web services. Besides, there's already a lot of open web services available to use, shortening the development time.

With the grown importance of those compositions technics, this work will study a little of the differences between them and also evaluate large scale synthetical compositions. This evaluation assist the development of the project \emph{Baile}, Institute of Mathematics and Statistics from University of São Paulo (IME-USP) in partnership with Hewlett-Packard (HP) of the United States, as well as the project \emph{CHOReOS}, IME-USP in partnership with a European consortium.

Baile’s main goal is to study the problems related to the development of choreography in large scale environments, particularly in the context of Cloud Computing. The ultimate goal of CHOReOS, for USP, will be to implement a Service Oriented Middleware for the ``Internet of the Future'', an \emph{Ultra-Large Scale internet}, heterogeneous and dynamic.

Also, by creating those compositions, this work will help in studying the problems for Baile and show how scalable the current techniques of choreography and orchestration are, later helping CHOReOS to understand the current state of the art of these techniques of composition of web services.