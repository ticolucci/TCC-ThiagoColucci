\section{Introduction}

With the development of the Internet, several Web Applications (such as Gmail, Google Docs, and Amazon.com) became very popular. Those applications use many web services for decoupling its parts. With plenty of services to use, we need some technic to compose those web services into new ones with a higher level of abstraction. In this context arose the Web service Orchestration and Choreography, technics to create services using only description languages and pre-existing web services.

Some of the advantages of using web services compositions are: interoperability, because web services are platform independent, they tend to be more scalable, by the fact that they are loosely-coupled, and there are already several open web services available to use, shortening the development time.

The growing importance of those composition technics motivated this work, we will study the differences between them and also evaluate large scale synthetical compositions. This evaluation will assist the development of the \emph{Baile project}, developed at the Institute of Mathematics and Statistics from the University of São Paulo (IME-USP) in partnership with Hewlett-Packard (HP) of the United States, as well as the project \emph{CHOReOS}, in which IME-USP is in partnership with an European consortium.

Baile’s main goal is to study the problems related to the development of choreographies in large scale environments, particularly in the context of Cloud Computing. The main objective of CHOReOS, for USP, will be the development of a Service Oriented Middleware for the ``Internet of the Future'', an \emph{Ultra-Large Scale internet}, heterogeneous and dynamic.

Also, by creating those compositions, this work will help the study of problems for Baile and show how scalable the current technics of choreographies and orchestration are, later helping CHOReOS to understand the current state of the art of these technics of composition of web services.