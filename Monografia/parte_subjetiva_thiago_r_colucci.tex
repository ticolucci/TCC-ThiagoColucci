\documentclass[11pt,a4paper]{article}
\usepackage[brazil]{babel}
\usepackage[utf8]{inputenc}
\usepackage[T1]{fontenc}

\usepackage{syntonly}

\usepackage{anysize}
\marginsize{3cm}{2cm}{2cm}{1cm}


\title{Parte Subjetiva do TCC}
\author{Thiago Rodrigues Colucci\\\\Orientador: Fabio Kon}


\pagestyle{headings}


\begin{document}
\maketitle

\newpage

``Cai de para-quedas'' no IME. Após estudar o ensino médio inteiro em Uberlândia, Minas Gerais, voltei para São Paulo para fazer cursinho e tentar entrar na USP, em engenharia mecânica. Eu ainda não sabia exatamente como seria a engenharia, mas, como sempre gostei de exatas, essa foi a escolha óbvia. Porém só consegui entrar de segunda chamada na minha segunda opção: Ciências da Computação. Resolvi ``experimentar'' o curso, na pior das hipóteses eu teria que fazer transferência ou prestar FUVEST novamente. Mas nada disso foi preciso. No IME descobri que Computação era o que sempre quis fazer e não sabia. Hoje só tenho a agradecer este instituto que me fez perder muitos cabelos, mas que também me ajudou muito no meu aprendizado, tanto acadêmico e profissional, quanto pessoal.

Este trabalho começou no primeiro semestre deste ano (2010), quando conversei com o Prof. Fabio Kon sobre projetos para mestrado ou doutorado direto. O Professor tinha dois projetos sobre composições de serviços web, e eu nunca tinha estudado nada sobre o assunto. Li um artigo e procurei mais informações na internet, acabei gostando do tema. Entrei para o projeto CHOReOS, e consequentemente para o projeto Baile, portanto a escolha do tema para o TCC foi impulsionada diretamente pelos objetivos destes projetos.





\section{ Desafios e frustrações encontradas}
Durante a graduação tive muitos desafios e frustrações, como as provas de Álgebra II e Cálculo III, que, não importando quantas horas de estudos nos dedicássemos, a nota era sempre pouco acima do mínimo necessário para a aprovação. Tais matérias criaram um lema no BCC 2007: ``O IME não te prepara nem para o mercado, nem para a o meio acadêmico. Somos sim preparados a superar qualquer desafio.'' Este é o sentimento que tínhamos, e é isto que me fez terminar este TCC. 

Descobri que fazer algo novo e, principalmente, sem estudo prévio é muito mais difícil do que acreditava ser. As estimativas nunca batem com o tempo gasto de fato, o software, que todas os fóruns dizem funcionar, na realidade nunca faz exatamente o esperado logo na primeira tentativa e a teoria as vezes parece muito simples, porém na prática não é bem assim. Tive que aprender sozinho a teoria e as tecnologias usadas neste trabalho.

Este sentimento de solidão me deixou muito irritado com o BCC, afinal de contas achei que estava me formando e não conseguiria entrar neste nicho se desejasse. No entanto o que aconteceu foi que percebi que conseguia aprender e assimilar as abstrações de forma bem rápida (porém não foi rápido o suficiente), e isto devo às matérias de matemática que sempre exigiram uma abstração ``acima do normal''. Confirmei então que o IME nos prepara sim para o mercado e para o meio acadêmico, mas faz isto nos ``ensinando a aprender''.

\section{Lista das disciplinas cursadas no BCC mais relevantes para o trabalho}
Creio que todas disciplinas influenciaram na construção da pessoa que sou hoje. Assim como cada uma influenciou um pouco na minha formação, o mesmo aconteceu no desenvolvimento deste trabalho. Porém para citar as disciplinas que me ajudaram diretamente, tenho poucas:

\begin{itemize}
	\item \textbf{MAC0110 - Introdução à Computação} - me ensinou o que era programação e como desenvolver o pensamento lógico para criação de um programa.
	\item \textbf{MAC0122 - Princípios de Desenvolvimento de Algoritmos} - aprimorou as técnicas de programação e fortaleceu mais o pensamento de programador.
	\item \textbf{MAC0323 - Estrutura de Dados} - fez com que entendesse melhor o que eram e como manipular as estruturas de dados de um programa, junto de MAC0110 e MAC0122, este trio consolida a base para conseguirmos superar e assimilar os próximos cursos da graduação.
\end{itemize}

Infelizmente não consigo pensar em mais nenhuma matéria que tenha me auxiliado diretamente. Este trabalho é basicamente programação para Web e a única matéria (com o mesmo nome da área) não foi oferecida num semestre em que eu tivesse disponibilidade de cursá-la. No entanto tive a sorte de participar de eventos como FISL, Rails Summit e Agile Conference e também de grupos como o \emph{Coding Dojo}, que me auxiliaram a suprir esta ``deficiência'' do curso.






\section{Futuro do projeto}

Os projetos Baile e CHOReOS estão apenas no começo de suas ``vidas'', ambos possuem menos de 1 ano. Então pretendo continuar as pesquisas no doutorado direto, desenvolvendo ainda mais estes projetos, tanto finalizando os experimentos deste TCC, quanto desenvolvendo novos sistemas. Este trabalho foi excelente para iniciar os estudos que realizaremos nesta área recente da Computação Distribuída (as Composições de Serviços Web) e também auxiliou no aprendizado de como criar estas Composições.




\section{Agradecimentos}

Não consigo olhar para trás e não mencionar e agradecer às pessoas que me ajudaram a completar a graduação. Primeiramente meus pais e meu irmão: Miguel, Angela e Lucas; sem o auxílio e inspiração deles, eu não seria um centésimo da pessoa que sou hoje. Meus melhores amigos durante a graduação (e que tenho certeza que ficarão para a vida toda): João Miranda, Lucianna Almeida e Pedro Leal; se não fosse o apoio nas noites que passamos estudando e os fins de semana que ``perdemos'' fazendo EP, hoje estaria perdido no meio de alguma matéria de matemática. Meus veteranos: Cecilia Fernandes, Fabricio Nascimento, Hugo Corbucci, Mariana Bravo e Danilo Sato; pessoas estas que me ensinaram a \emph{codar} e me trouxeram para o mundo de sistemas. Meu orientador e Professor: Fabio Kon; que realmente me guiou para completar este TCC. Finalmente, mas não menos importante, o pessoal dos projetos Baile e CHOReOs, que me ajudaram a entender este mundo tão diferente da Web.

Muito obrigado a todos!

\end{document}