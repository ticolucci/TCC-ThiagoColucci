\documentclass[12pt,a4paper]{article}
\usepackage[brazil]{babel}
\usepackage[utf8]{inputenc}
\usepackage[T1]{fontenc}

\usepackage{syntonly}


\title{Seminar\\Web Services Orchestration}
\author{Thiago Rodrigues Colucci}

\pagestyle{headings}

%\syntaxonly

\usepackage[pdftex,bookmarks,colorlinks]{hyperref}

\begin{document}
\maketitle

\newpage

\tableofcontents

\newpage


\section{Web Services}
\subsection{What are Web Services?}
\begin{itemize}
\item Web services are application components
\item Web services communicate using open protocols
\item Web services are self-contained and self-describing
\item Web services can be discovered using UDDI
\item Web services can be used by other applications
\item XML is the basis for Web services
\end{itemize}

\subsection{How Does it Work?}
The basic Web services platform is XML + HTTP.

XML provides a language which can be used between different platforms and programming languages and still express complex messages and functions.

The HTTP protocol is the most used Internet protocol.

Web services platform elements:
\begin{itemize}
\item SOAP (Simple Object Access Protocol)
\item UDDI (Universal Description, Discovery and Integration)
\item WSDL (Web Services Description Language)
\end{itemize}

\subsection{Why Web Services?}

\subsubsection{Interoperability has Highest Priority}

When all major platforms could access the Web using Web browsers, different platforms could interact. For these platforms to work together, Web-applications were developed.

Web-applications are simple applications that run on the web. These are built around the Web browser standards and can be used by any browser on any platform.

\subsubsection{Web Services take Web-applications to the Next Level}

By using Web services, your application can publish its function or message to the rest of the world.

Web services use XML to code and to decode data, and SOAP to transport it (using open protocols).

With Web services, your accounting department's Win 2k server's billing system can connect with your IT supplier's UNIX server.

\subsubsection{Web Services have Two Types of Uses}

\bf{Reusable application-components.}

There are things applications need very often. So why make these over and over again?

Web services can offer application-components like: currency conversion, weather reports, or even language translation as services.

\bf{Connect existing software.}

Web services can help to solve the interoperability problem by giving different applications a way to link their data.

With Web services you can exchange data between different applications and different platforms.









\section{Bibliography}
\begin{itemize}
\item Web Service Tutorial - http://www.w3schools.com/webservices/
\item C. Peltz. Web services orchestration: a review of emerging technologies, tools, and standards. Technical Report HP. January, 2003 - http://xml.coverpages.org/HP-WSOrchestration.pdf
\item Non-intrusive monitoring and service adaptation for WS-BPEL. 17th international conference on World Wide Web, 2008 - http://doi.acm.org/10.1145/1367497.1367607
\item 
\end{itemize}


\end{document}