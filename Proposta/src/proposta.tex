\documentclass[11pt,a4paper]{article}
\usepackage[brazil]{babel}
\usepackage[utf8]{inputenc}
\usepackage[T1]{fontenc}

\usepackage{syntonly}

\usepackage{anysize}
\marginsize{3cm}{2cm}{2cm}{1cm}


\title{Proposta de Monografia}
\author{Thiago Rodrigues Colucci\\\\Prof. Dr. Fabio Kon}


\pagestyle{headings}

%\syntaxonly

\begin{document}
\maketitle

\newpage

\tableofcontents
\newpage
\textbf{Aluno:} Thiago Rodrigues Colucci

\textbf{Supervisor:} Prof. Dr. Fabio Kon

\section{Tema}
Implementação de orquestrações e coreografias de serviços Web que serão utilizados para realizar a avaliação da escalabilidade de tais tipos de serviços em ambientes de grande escala, neste caso nas nuvens da Amazon \cite{amazonEC2}.

\section{Resumo da monografia}
O trabalho auxiliará no desenvolvimento do projeto Baile, do Instituto de Matemática e Estatística da Universidade de São Paulo (IME-USP) em parceria com a Hewlett-Packard (HP) dos Estados Unidos, bem como no projeto CHOReOS, do IME-USP em parceria com um consórcio europeu. 

O Baile tem como objetivo estudar os problemas relacionados ao desenvolvimento de coreografias em ambientes de grande escala, em particular no contexto de Computação em Nuvem. O maior objetivo do CHOReOS, para a USP, será a implementação de um Middleware Orientado a Serviço para a ``Internet do Futuro'', uma Internet com escala ultra grande (\textit{Ultra-Large Scale}), heterogênea e dinâmica. 

Este trabalho ajudará nos estudos dos problemas para o Baile e também mostrará quão escalável as técnicas atuais de coreografia e orquestração são, auxiliando o CHOReOS a entender o atual estado da arte destas técnicas de composição de serviços.

\section{Objetivos}
O maior objetivo do trabalho é realizar experimentos preliminares de avaliação de desempenho de composições de serviços web e sua escalabilidade nas nuvens da Amazon. Com isto poderemos prosseguir, nos próximos anos, o desenvolvimento do Middleware para o CHOReOS e também especificar melhor os métodos e arcabouços para Verificação e Validação de coreografias de grande escala.

Para efetuar tais experimentos, será necessário o desenvolvimento de um pequeno arcabouço que permita a realização de experimentos de grande escala de orquestração e coreografias. 

As composições utilizadas nos testes serão geradas automaticamente. No caso das orquestrações serão simples árvores com uma quantidade aleatória de níveis e quantidades aleatórias de filhos em cada nó. Já as coreografias serão modeladas como \textit{workflows}.

\section{Atividades já realizadas}
No início de março comecei os estudos sobre Serviços Web e outras tecnologias de base para o trabalho, como SOAP, REST, WSDL entre outras. Depois foquei no estudo sobre o que são orquestrações e coreografias e, principalmente, como devem ser implementadas para atingir uma grande escalabilidade. Neste ponto foi difícil achar muitas literaturas diferentes sobre o assunto, isso porque na década de 2000 a área ficou um pouco abandonada, voltando a ganhar força na acadêmia e na indústria.

Ao longo de abril preparei um seminário, apresentado no início de maio, sobre orquestrações de serviços web. De maio até junho, momento atual, estou rodando testes preliminares que, ao longo dos próximos meses, evoluirão para o arcabouço.

\section{Cronograma}
Tentarei seguir o seguinte cronograma:

\begin{table}[h]
  \begin{center}
    \begin{tabular}{ | p{5.2cm} || c | c | c | c | c | c | c | c | c |}
      \hline
      \textbf{Atividade / Mês}              & \textbf{Mar} & \textbf{Abr} & \textbf{Mai} & \textbf{Jun} & \textbf{Jul} & \textbf{Ago} & \textbf{Set} & \textbf{Out} & \textbf{Nov} \\
      \hline
      \hline
      Estudo da tecnologia                              &  X  &  X  &     &     &     &     &     &     &     \\
      \hline
      Desenvolvimento do arcabouço                      &     &     &  X  &  X  &  X  &  X  &  X  &     &     \\
      \hline
      Monografia                                        &     &     &     &     &  X  &  X  &  X  &  X  &     \\
%      \hline
%      Submissão de artigo para workshop de Middleware * &     &     &     &     &     &  X  &     &     &     \\
      \hline
      Execução dos experimentos                         &     &     &     &     &     &  X  &  X  &  X  &     \\
      \hline
      Apresentação                                      &     &     &     &     &     &     &     &     &  X  \\
      \hline
    \end{tabular}
  \end{center}
  \caption{* porém não temos certeza se teremos os resultados antes do prazo limite para submissão}
\end{table}

\section{Estrutura esperada da monografia}
\subsection{Parte técnica}
\begin{enumerate}
\item \textbf{Introdução:}  \\ Breve explicação dos motivos de se fazer composição de serviços web.
\item \textbf{Conceitos e tecnologias estudadas:}  \\ Explicação do que são serviços web, como os implementamos, definições de orquestração, coreografias, computação nas nuvens e a ``Internet do Futuro''. Com isto o leitor compreenderá as demais seções sem dificuldades.
\item \textbf{Atividades realizadas:}  \\ Descrição de como foram criadas as composições de serviços web e experimentos realizados no processo inteiro.
\item \textbf{Resultados e produtos obtidos:}  \\ Nesta seção estarão, basicamente, os resultados dos testes.
\item \textbf{Conclusões}. 
\item \textbf{Bibliografia.} 
\end{enumerate}


\subsection{Parte subjetiva}
\begin{enumerate}
\item desafios e frustrações encontrados;
\item lista das disciplinas cursadas no BCC mais relevantes para o trabalho;
\item observações sobre a aplicação de conceitos estudados nas disciplinas do curso;
\item se o aluno fosse continuar atuando na área em que realizou o trabalho, que passos tomaria para aprimorar os conhecimentos relevantes para esta atividade?
\end {enumerate}

\end{document}